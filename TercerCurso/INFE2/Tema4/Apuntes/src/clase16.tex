\section{Introsucción al test de Bondad de Ajuste}

Hasta ahora los problemas planteados parten de unos datos obtenidos
en un experimento aleatorio del que se conoce el mecanismo con el que se genera (Familia de distribución $P_\theta$).

\textit{\textbf{Definición: }}El Test de Bondad de Ajuste es un test para comprobar si una familia de distribuciones,
 representa correctamente el mecanismo con el que se generaron los datos.

Planteamiento:

$X_1,\dots,X_n$ con funcion de distribución F. Si $F_0$ es una función de distribución conocida, (por ejemplo, una Poisson con $\lambda=3$), el problema se reduce a:

$$H_0: F=F_0 \quad H_1: F \neq F_0$$
$F_0$ está completamente especificada, por lo que es una hipótesis simple.

La hipótesis nula es compuesta si $F_0$ depende de parámetros desconocidos, por ejemplo, una P($\lambda$)

%Posible section: H_0 simple.
Empezamos con el caso de $H_0$ simple. Existen dos tipos de Test de Bondad de Ajuste si $F_0$ es una función de  distribución conocida.
\begin{enumerate}
    \item $F_0$ discreta: Se comparan las frecuencias observadas con las esperadas bajo $H_0$ (Test $\chi^2$). También se puede hacer si $F_0$ es continua agrupando, pero hay tests más potentes para esos casos.
    \item $F_0$ continua: Comparamos la funcion de distribución empírica con la teórica.
\end{enumerate}

\subsection{Test Chi-Cuadrado de Bondad de Ajuste} 

Sea $X_1,\dots,X_n$ con distribución F discreta:
\[
\begin{matrix}
    C_1 \to P_1\\
    \dots \\
    C_k \to P_k
\end{matrix}
\quad
p_j \geq 0,
\quad \sum_{j=1}^{n} p_j=1
\]

Caso en el que $F_0$ completamente especificada bajo $H_0$:

\[
H_0: p=p^0 \quad p=(p_1,\dots,p_k)'
\]
\[
H_1: p \neq p^0 \quad p^0=(p_1^0,\dots,p_k^0)'
\]

Este problema ya lo sabemos resolver: es un contraste para el parámetro p de una distribución multinomial.

Frecuencia observada en $C_j: f_j=\sum_{i=1}^{n} \mathbbm{1}_{(x_i=c_j)}, \quad j=1,\dots,k \quad \sum_{j=1}^{k} f_j=n$

Frecuencia esperada bajo $H_0$ en $C_j$:
\[
e_j=n \cdot P_0(x=c_j)=n \cdot p_j^0
\]
\newpage
\textbf{Ejemplo:}

\[
P(X = x) = 
\left\{
    \begin{matrix}
        \frac{1}{3} & \text{si } x = 0 \\[0.5em]
        \frac{1}{3} & \text{si } x = 1 \\[0.5em]
        \frac{1}{3} & \text{si } x = 2
    \end{matrix}
\right.
\]


\[
H_0:\quad p_1=\frac{1}{3} \quad p_2=\frac{1}{3} \quad p_3=\frac{1}{3}
\]

Tomando una muestra de n=10. Me salen 7 ceros, 2 unos y 1 dos.
A simple vista no parece que siga esa distribución. Usaremos un estadístico para medir lo diferente que es de nuestra distribución.

Lo que esperabamos obtener es:
\[
\begin{matrix}
    e_1=3,33\\
    e_2=3,33\\
    e_3=3,33
\end{matrix}
\]

El test $\chi^2$ de ajuste proporciona unas bases probabilisticas para decidir si las diferencias son suficientemente grandes para que no hayan ocurrido por azar.

El estadístico se define como:

\[
\chi^2=\sum_{j=1}^{k} \frac{(f_j-e_j)^2}{e_j}
\]

Este estadístico es lo que se conoce como distancia $\chi^2$ entre $f_j$ y $e_j$.

Valores grandes en $\chi^2$, frecuencias observadas y esperadas muy diferentes. 
Fijado $\alpha$:
 \begin{itemize}
    \item  Región critica: ($\chi^2 > C_\alpha$)
    \item  p-valor: $p_0(\chi^2>X_{obs})=p_0(\chi^2>t_{obs})$
 \end{itemize}

 Necesitamos conocer la distribución del estadístico bajo $H_0$.
 Se puede conocer de forma exacta aunque es muy complejo. Nos interesará la distribución asintótica para n grande.

 \textbf{Resultado:}
 \[
 \lim_{n \to \infty} P_0(\chi^2 \leq Z)=P(\chi^2_{k-1} \leq Z)
 \]

La distribución asintótica del estadístico $\chi^2$ bajo $H_0$ es $\chi^2_{k-1}$ g'.

Falla el proofs
\begin{proof}
Ya hemos visto que en este contexto \( F_0 \) es una distribución multinomial. 
Vamos a ver que el estadístico \( T \) bajo \( H_0 \) para el modelo multinomial con \( k-1 \) parámetros libres es asintóticamente equivalente al estadístico \( \chi^2_1 \).

Estadístico RV para:

\[
H_0: F = F_0
\]
\[
H_1: F \neq F_0
\]
\newpage
Recordemos:

\[
L(p, x) = \prod_{j=1}^{k} p_j^{f_j} \quad \implies \quad \log L(p, x) = \sum_{j=1}^{k} f_j \log p_j
\]

El estadístico es:

\[
T = 2 \left[\log L(\hat{p}, x) - \log L(p_0, x)\right] = 2 \left[ \sum_{j=1}^{k} f_j \log \hat{p_j} - \sum_{j=1}^{k} f_j \log p_j^0 \right] 
\]\[= -2 \sum_{j=1}^{k} \left( \log p_j^0 - \log \hat{p_j} \right)
\]

Aproximamos \( \log p_j^0 \) por un desarrollo de Taylor en torno a \( \log \hat{p_j} \):

\[
\log p_j^0 \approx \log \hat{p_j} + (p_j^0 - \hat{p_j}) \frac{1}{\hat{p_j}} - \frac{(p_j^0 - \hat{p_j})^2}{2 \cdot \hat{p_j}^2}+ \dots
\]
\[
\to \log p_j^0- \log \hat{p_j}\thickapprox (p^0_j-\hat{p_j}) \frac{1}{\hat{p_j}}- \frac{(p^0_j-\hat{p_j})^2}{2}\frac{1}{\hat{p_j^2}}
\]
Sabiendo que $\hat{p_j}=\frac{f_j}{n}$
\[
=\left(p_j^0-\frac{f_j}{n}\right) \cdot \frac{n}{f_j} - \frac{\left(p_j^0-\frac{f_j}{n}\right)^2}{2}\cdot \left(\frac{n}{f_j}\right)^2 \xrightarrow{c.s.} 0
\]

Esto converge a 0, ya que, por la Ley de los Grandes Números (LGN), \( \frac{F_j}{n} \) es un estimador consistente de \( p_j \):

\[
\lim_{n \to \infty} P \left( \left| \frac{F_j}{n} - p_j \right| > \varepsilon \right) = 0 \quad \forall \varepsilon>0
\]

Por lo tanto, el estadístico T queda:

\[
T = -2 \sum_{j=1}^{k} \left( n \cdot p_j^0 - f_j \right) + \frac{\sum_{j=1}^{k} (f_j - n \cdot p_j^0)^2}{f_j}
\]

\end{proof}


%Falla el proofs

El text $\chi^2$ y el TRV son asintóticamente equivalentes y como TRV converge a $\chi^2_{k-1}$, el estadístico $\chi^2$ también.
Fijado un $\alpha$.
\begin{itemize}
    \item Región critica: $P_0(\chi^2 \geq C_\alpha) \quad C_\alpha=qchisq(1-\alpha,k-1)$
    \item p-valor: $P_0(\chi^2 \geq t_{obs})=1-qchisq(t_{obs},k-1)$
\end{itemize}

Nota:

Esta aproximación es válida para frecuencias mayores o iguales a 5

\textbf{Ejercicio 1:} Script R

\textbf{Ejercicio 4:} En una fábrica con 220 empleados, el número de trabajadores que tuvieron accidentes se recoge en latabla siguiente:
\[
\begin{matrix}
    \text{Nº accidentes} & 0& 1& 2& 3&  4& 5& 6+\\
    \text{Nº trabajadores}&  181& 9& 4&10&7&4&5
\end{matrix}
\]
¿Son estos datos consistentes con la distribución de Poisson con $\lambda=1$?

Tenemos una hipótesis compuesta ya que no conocemos el parámetro de la Poisson.
Si nos dijeran $P(1)$ sería simple.

\[
f(x,1)=\frac{e^{-1}\cdot \lambda^x}{x!}=\frac{e^{-1}}{x!}
\]
\[
p_0^0=e^{-1} ,\quad p_1^0=e^{-1} ,\quad p_2^0=\frac{e^-1}{2} ,\quad \dots,\quad p_6^0=1-\sum_{j=0}^{b}p_j^0
\]
Calculamos las frecuencias esperadas
\[
e_0=220 \cdot e^{-1},\quad e_1=220 \cdot e^{-1}, \quad \dots
\]
\[
\chi^2=\frac{(181-220 \cdot e^{-1})^2}{220 \cdot e^{-1}}+\dots
\]

Es muy probable que no siga una P(1)

\textbf{Ejercicio 10:} Se lanza una moneda hasta que aparece la primera cara. Este experimento se repite 100 veces. LAs frecuencias observadas del numero de ensayos necesarios hasta que aparece la primera cara son:
\[
\begin{matrix}
    \text{Nº ensayos} & 1& 2& 3& 4&5+\\
    \text{Frecuencia} &40&32&15&7& 6
\end{matrix}
\]

¿Se puede concluir que la moneda es perfecta?

Distribución geométrica:
\[
P_p(X=k)=(1-p)^{k-1}p \quad o \quad P_p(X=k)=(1-p)^{k}p
\]
Probabilidad bajo $H_0$.
\[
p_1^0=\frac{1}{2} \quad p_2^0=\frac{1}{2^2} \quad \dots \quad p_5=1-\sum_{k=1}^{4} p_k
\]



 