\subsubsection{Modelo poblacional}

El precio que pagamos usando un modelo de aleatorización es que los resultados solo son válidos para los N individuos de estudio y no se pueden extrapolar a una población más amplia.
Para que eso sea posible, será necesario que los N individuos representen a toda la población. Dicho de otra forma, necesitamos una \textbf{muestra aleatoria simple} de la población.

La situación es la siguiente:
Tenemos $N=n+m$ individuos al azar de la población,

$$
\begin{aligned}
    n & \longrightarrow \text{elegidos al azar}\longrightarrow \text{grupo de tratamiento} \\
    m & \longrightarrow \text{restantes al grupo de control} \\
    \ & \ \\
    Y: & \text{ Variable respuesta de individuos que reciben el tratamiento}\\
    X: & \text{ Variable respuesta de individuos que son del grupo de control}  
\end{aligned}
$$

X e Y son dos variables aleatorias con funciones de distribución $X\sim F$ y $Y\sim G$ \\
\noindent Queremos contrastar la hipótesis de que el tratamiento NO es efectivo

$$
\begin{aligned}
    H_0: & \text{ El tratamiento no tiene efecto (F=G)} \\
    H_1: & \text{ El tratamiento aumenta/disminuye la respuesta (F>G/F<G)}  
\end{aligned}
$$

El modelo poblacional tiene dos ventajas fundamentales:
\begin{enumerate}
    \item Los resultados son extrapolables
    \item Podemos estudiar la potencia del test
\end{enumerate}

\noindent Si tenemos un modelo poblacional sin coincidencias podemos utilizar el estadístico $W_s$ y el test de Wilcoxon $(W_s>C_\alpha)$; bajo $H_0$, $W_s$ sigue la misma distribución que en el modelo de aleatorización.

$$
\begin{aligned}
    X_1,\dots, X_m & \quad Y_1,\dots,Y_n\\
    R_1,\dots, R_m & \quad S_1,\dots,S_n
\end{aligned}
$$
$$
W_s=S_1+\dots+S_n
$$

Si hay coincidencias, tenemos que encontrar la distribución de $W^*_s$ bajo $H_0$. \\
En este caso el estadístico $W^*_s=S^*_1+\dots+S^*_n$ no es de distribución libre. La distribución bajo $H_0$ de los semi-rangos de los n individuos depende de F. 
Esto se debe (al igual que en el modelo de aleatorización) a que la distribución depende de la configuración de las coincidencias $(e,d_1,\dots,d_e)$, que en el modelo de aleatorización son un número pero aquí son variables aleatorias cuya distribución depende de F.

\begin{theorem}
    Supongamos F discreta de tal forma que 
    
    $$
        F:\begin{cases}
            a & \text{Con probabilidad } p \\
            b & \text{Con probabilidad } 1-p
        \end{cases}
    $$
    Si a<b, y con m=2 y n=1, entonces los posibles resultados son:

    $$
    \begin{array}{c|c|c}
        X_1X_2Y_1 & \text{Probabilidad} & \text{Semi-rangos} \\ \hline
        a \; a \; a & p^3 & 2 \quad 2 \quad 2 \\ 
        a \; a \; b & p^2(1-p) & 1.5 \quad 1.5 \quad 2 \\ 
        a \; b \; a & p^2(1-p) & 1.5 \quad 2 \quad 1.5 \\ 
        b \; a \; a & p^2(1-p) & 2 \quad 1.5 \quad 1.5 \\ 
        a \; b \; b & p(1-p)^2 & 1 \quad 2 \quad 2 \\ 
        b \; b \; a & p(1-p)^2 & 2 \quad 2 \quad 1 \\ 
        b \; a \; b & p(1-p)^2 & 1.5 \quad 1 \quad 1.5 \\ 
        b \; b \; b & (1-p)^3 & 1 \quad 1 \quad 1 \\ 
    \end{array}
    $$
    La distribución de $W^*_s$ bajo $H_0$ será:

    $$
    \begin{array}{c|c c c c c}
        S^*_n & 1 & 1.5 & 2 & 2.5 & 3 \\ \hline 
        P_0(S^*_1=s^*_1) & p(1-p)^2 & 2p^2(1-p) & p^3+(1-p)^3 & 2p(1-p)^2 & p^2(1-p) \\
    \end{array}
    $$

\end{theorem}


