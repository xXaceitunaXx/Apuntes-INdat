\section{Test de rangos}

Caso paramétrico normal.

$X_1,\dots,X_n$ i.i.d. $N(\mu_1,\sigma_1)$

$Y_1,\dots,Y_n$ i.i.d. $N(\mu_2,\sigma_2)$

\[
    H_0:\mu_1=\mu_2
\]
\[
    H_1: \mu_1 \neq \mu_2
\]

En el caso no paramétrico:

$X_1,\dots,X_n$ i.i.d. con distribución $F$

$Y_1,\dots,Y_n$ i.i.d. con distribución $G$

\[
    H_0:F=G
\]
\[
    H_1: F \neq G
\]

En ambos casos el objetico es el mismo, comparar tratamientos o resultados.

\subsubsection*{Ejemplo:} 
Digamos que se quiere contrastar la eficacia de un nuevo medicamento para una enfermedad.
Lo primero que se tiene que hacer es diseñar un experimento para obtener los datos.

\begin{center}
    \href{https://media.tenor.com/lsbTX_Avt2AAAAAM/brushing-teeth-kowalski.gif}{Kowalski}, opciones (para diseñar el experimento xd):
\end{center}

\noindent Tenemos dos opciones...
\begin{enumerate}
    \item \textbf{Modelo de aleatorización}: Los datos vienen de un diseño controlado en el que los 
    individuos de análisis han sido asignados aleatoriamente a diferentes grupos.
    \item \textbf{Modelo poblacional:} Los datos son extraidos de una población y se asume que esa población tiene ciertas propiedades. 
    Por ejemplo: una distribución
\end{enumerate}

\subsection{Modelo de aleatorización}

Se dispone de N individuos. Se eligen n individuos a los que se asigna un tratamiento(una medición).
Caso donde

\[
    X_1,\dots,X_m \quad H_0: \text{El tratamiento no tiene efecto} 
\]
\[
    Y_1,\dots,Y_n \quad H_1: \text{El tratamiento tiene efecto} 
\]

El estadístico que utilicemos, rechazará $H_0$ cuando los valores de la variable considerada en los tratados sean mayores a los de control.
Para esto, usaremos \textbf{el estadístico basado en rangos}.
El estadístico basado en rangos, no depende de unidades de medida. Tomaremos toda la muestra y asignaremos rangos.

\textbf{\textit{Definición:}} Un rango es el lugar que ocupa la observación en la muestra ordenada.
Sean ($X_1,\dots,X_m$, $Y_1,\dots,Y_n$) nuestra muestra completa. Se ordenan los valores y se asignan rangos ($i_1,\dots,i_m,i_{m_1},i_n$)
Esto es la permutación donde $i_k$ es el valor que ocupa la observación k-ésima en la muestra ordenada.

\subsubsection*{Ejemplo:}

Tenemos $X=\{5,8,9\}$ e $Y=\{6,7,10\}$

Por tanto la muestra completa será $\{5,8,9,6,7,10\}$

Que ordenado $\{5,6,7,8,9,10\}$, y se le asigna un rango a cada elemento ($\{1,2,3,4,5,6\}$)

\vspace{2mm}

\noindent \textit{Usando notación de tuplas, ($a, b$) donde $a$ es un elemento de la muestra y $b$ el rango asociado, la anterior asignación quedaría de la siguiente manera $\{(5, 1),(6, 2),(7, 3),(8, 4),(9, 5),(10, 6)\}$}

\vspace{2mm}

\noindent Por tanto quedarían asignados...

Rangos de X: $\{1,4,5\}$.

Rangos de Y: $\{2,3,6\}$.

\vspace{5mm}

\noindent \textbf{Notación formal:}

$R_1,\dots,R_m \longrightarrow$ rangos correspondientes a las observaciones $X_1,\dots,X_m$.

$S_1,\dots,S_n \longrightarrow$ rangos correspondientes a las observaciones $Y_1,\dots,Y_n$.

¿Cuando las $Y_s$ son mayores que las $X_s$?. Cuando $s_i$ sean más grandes, es decir, la suma de rangos sea más grande.

\subsubsection{Estadístico de suma de rangos}
\textit{\textbf{Definición: }} La suma
\[
    W_s=s_1+\dots+s_n
\]
es conocida como \href{https://es.wikipedia.org/wiki/Prueba_de_los_rangos_con_signo_de_Wilcoxon}{el estadístico de Wilcoxon} de suma de rangos ($W_r=R_1+\dots+R_m$). Se rechazará $H_0$ para valores de $W_s$ grandes ($W_s>c_\alpha$).

Como siempre, debemos conocer la distribución del estadístico de $W_s$ bajo $H_0$. Como la distribución es discreta, y podemos calcular lo siguiente...
\[
    P_{H_0}(W_s=k)=\sum_{s_1+\dots+s_n=k} P_{H_0} ((S_1,\dots,S_n)=(s_1,\dots,s_n))
\]

... encontrar la distribución de $W_s$ bajo $H_0$ se reduce a encontrar la distribución de ($S_1,\dots,S_n$).

\[
    P_{H_0} ((S_1,\dots,S_n)=(s_1,\dots,s_n))=\frac{1}{\binom{N}{n}}
\]
Cada resultado es igual de probable bajo $H_0$.

\newpage

\subsubsection*{Ejemplo:}
\noindent Dados $N=5, m=2, n=3$

\[
    \binom{N}{n}=\binom{5}{3}=\frac{5!}{3!\cdot2!}=10
\]
Habrá 10 posibles resultados de $s_1,s_2,s_3$.
\[
    \begin{array}{|c|c|c|c|c|c|}
    \hline
    \textbf{Tratados} & (1,2,3) & (1,2,4) & (1,2,5) & \dots & (3,4,5) \\ \hline
    P_{H_0}((S_1, \dots, S_n) = (s_1, \dots, s_n)) & 0.1 & 0.1 & 0.1 & \dots & 0.1 \\ \hline
    W_s & 6 &7 &8 & \dots &12 \\ \hline
    \end{array}
\]

\noindent Distribución de \(W_s\) bajo \(H_0\):
\[
    \begin{array}{|c|c|c|c|c|c|c|c|}
    \hline
    k & 6 & 7 & 8 & 9 & 10 & 11 & 12 \\ \hline
    P_{H_0}(W_s = k) & 0.1 & 0.1 & 0.2 & 0.2 & 0.2 & 0.1 & 0.1 \\ \hline
    \end{array}
\]

\noindent Rechazaremos $H_0$ cuando el valor de $W_s$ sea poco probable.

\vspace{2mm}

\noindent \textbf{Observaciones:}

La distribución de $W_s$ no es la misma si hubiéramos asignado n a tratamientos y n a control. Lo que es lo mismo, $W_R$ no sigue la misma distribución que $W_s$ bajo $H_0$. La distribución dependerá de n y m.

\subsection{Estadístico de Mann-Whitney}

En el caso anterior, el valor mínimo de $W_s$ corresponde a la situaciñon a la que los individuos con tratamiento toman los valores más pequeños. Donde en el ejemplo anterior, $W_s=6 \quad 6=\frac{n \cdot (n+1)}{2}$.
Si consideramos el estadístico de Mann-Whitney,
\[
    W_{XY}=W_S-\frac{n(n+1)}{2}
\]
se nos facilitará hacer la tabla porque tomas valores $0,1,\dots,(n \cdot m)$. Tomará el valor 0 cuando todos los valores de Y son los más pequeños y el valor n $\cdot$ m  cuando todos los Y tomen los valores más grandes.

Una ventaja que tiene el estadístico es que toma los mismos valores sin importar la decisión de cuántos asignar a $n$ y cuántos a $m$.

Del mismo modo:
\[
    W_{YX}=W_R-\frac{m(m+1)}{2}
\]
$W_{YX}$ también toma valores 0,1,$\dots,n \cdot m$.

\noindent $W_{XY}$ y $W_{YX}$ siguen la misma distribución bajo $H_0$.

\newpage

\noindent Existen tablas para esta distribución.

\vspace{2mm}

\noindent \textbf{Observaciones:}

La distribución bajo $H_0$ de $W_S$ (o $W_R)$ es simétrica respecto a $\frac{n \cdot (N+1)}{2}$.
\[
    \forall k \quad P_{H_0} \left( W_S=\frac{n \cdot (N+1)}{2}+k\right)=
    P_{H_0} \left( W_S=\frac{n \cdot (N+1)}{2}-k\right)
\]
Bajo $H_0$ X e Y están igualmente distribuidas, por lo que todos los elementos deberían ser indistinguibles en término de rangos. Por esto $W_{XY}$ y $W_{YX}$ están igualmente distribuidas bajo $H_0$.

\subsubsection*{Ejemplo del uso de las tablas:}
\noindent Tenemos $W=10,n=6,m=4$

\noindent Queremos saber $P_{H_0}(W_s \geq 35)$

\noindent Tenemos las tablas para $W_{XY}=W_S-\frac{n \cdot (N+1)}{2}=W_S-21$

\noindent Debemos escribir $P_{H_0}(W_s \geq 35)$ como $P_{H_0}(W_{XY} \leq a)$

\noindent Sabemos que $W_S$ es simétrico a $\frac{n \cdot (N+1)}{2}=33$. Entonces...
\[
    P_{H_0}(W_S \geq 35)= P_{H_0}(W_S \geq 33+2)=P_{H_0}(W_S \leq 33-2)=P_{H_0}(W_S \leq 31)
\]
\[
    =P_{H_0}\left(W_S-\frac{n \cdot (n+1)}{2} \leq 31-21\right)= P_{H_0} (W_{XY} \leq 10)=0.3810 (\text{Usando las tablas})
\]
Llegaríamos al mismo resultado usando $W_S+W_R=55$.
\[
    P_{H_0}(W_S \geq 35)= P_{H_0}(W_R \leq 20)= P_{H_0} \left( W_R - \frac{m \cdot (m+1)}{2} \leq 20-10\right)
\]
\[
    = P_{H_0} (W_{YX} \leq 10)=0.3810
\]
Hay tablas hasta $n=m=10$. A partir de así usariamos la distribución asintótica.

\subsubsection{Forma alternativa del estadístico de Mann-Whitney}

Una manera alternativa de ver el estadístico de Mann-Whitney que va a ser útil para calcular la potencia y también intervalos de confianza es la siguiente

\textit{\textbf{Definición: }} Si $X_1,\dots,X_m$ son valores para individuos sin tratamientos y $Y_1,\dots,Y_n$ son valores para individuos con tratamiento,
y $W_{XY}=W_S-\frac{n \cdot (n+1)}{2}$, $W_{XY}$ es también el número de pares ($X_i,Y_j$) $i=1,\dots,m$, $j=1,\dots,n$ para los que $X_i < Y_i$
\[
    W_{XY}= \# [(X_i,Y_j) | X_i <Y_j]
\]

\begin{proofs}
    Sean $Y_{(1)}, \dots, Y_{(n)}$ valores ordenados de $Y_1,\dots,Y_n$ y sean $s_1,\dots,s_n$ los rangos correpondientes también ordenados,
    \begin{itemize}
        \item Hay $s_1-1$ observaciones menores que $Y_{(1)}$ y todas son X.
        \[
            \# [(X_i,Y_{(1)}) | X_i <Y_{(1)}]=S_1-1
        \]
        \item Hay $s_2-1$ observaciones menores que $Y_{(2)}$, de ellas una es $Y_{(1)}$ y el resto son X.
        \[
            \# [(X_i,Y_{(2)}) | X_i <Y_{(2)}]=S_2-1-1=S_2-2
        \]
        \item Hay $s_n-1$ observaciones menores que $Y_{(n)}$, de ellas n-1 son Y y el resto son X.
        \[
            \# [(X_i,Y_{(n)}) | X_i <Y_{(n)}]=S_n-n
        \]
    \end{itemize} 
    Por lo tanto:
    \[
        \# [(X_i,Y_j) | X_i <Y_j]=(S_1-1)+(S_2-2)+ \dots +(S_n-n)
    \]
    \[    
        =(S_1+\dots+S_n)-(1+2+\dots+n)=W_S-\frac{n \cdot (n+1)}{2}=W_{XY}
    \]
\end{proofs}

\vspace{5mm}

Distribución asintótica de $W_s$
Si los valores $n$ y $m$ son grandes (mayores que 10), se considera que la distribución asintótica para $W_S$ bajo $H_0$ es, por el Teorema Central del Limite...

\[
    \frac{W_S-E_0(W_S)}{\sqrt{Var_{0}(W_S)}} \xrightarrow[H_0]{L} N(0,1)
\]
\[
    E_0(W_S)=\frac{n \cdot (N+1)}{2} \quad Var_0(W_S)=\frac{n \cdot (N-n) \cdot (N+1)}{12}
\]