\documentclass{article}

\usepackage{amsmath, amsthm, amssymb, amsfonts}
\usepackage{thmtools}
\usepackage{graphicx}
\usepackage{setspace}
\usepackage{geometry}
\usepackage{float}
\usepackage{hyperref}
\usepackage[utf8]{inputenc}
\usepackage[english]{babel}
\usepackage{framed}
\usepackage[dvipsnames]{xcolor}
\usepackage{tcolorbox}

\colorlet{LightGray}{White!90!Periwinkle}
\colorlet{LightOrange}{Orange!15}
\colorlet{LightGreen}{Green!15}

\newcommand{\HRule}[1]{\rule{\linewidth}{#1}}

\declaretheoremstyle[name=Theorem,]{thmsty}
\declaretheorem[style=thmsty,numberwithin=section]{theorem}
\tcolorboxenvironment{theorem}{colback=LightGray}

\declaretheoremstyle[name=Proposition,]{prosty}
\declaretheorem[style=prosty,numberlike=theorem]{proposition}
\tcolorboxenvironment{proposition}{colback=LightOrange}

\declaretheoremstyle[name=Principle,]{prcpsty}
\declaretheorem[style=prcpsty,numberlike=theorem]{principle}
\tcolorboxenvironment{principle}{colback=LightGreen}

\setstretch{1.2}
\geometry{
    textheight=9in,
    textwidth=5.5in,
    top=1in,
    headheight=12pt,
    headsep=25pt,
    footskip=30pt
}

% ------------------------------------------------------------------------------

\begin{document}

% ------------------------------------------------------------------------------
% Cover Page and ToC
% ------------------------------------------------------------------------------

\title{ \normalsize \textsc{}
\\ [2.0cm]
\HRule{1.5pt} \\
\LARGE \textbf{\uppercase{Template Title}
\HRule{2.0pt} \\ [0.6cm] \LARGE{Very Cool Subtitle} \vspace*{10\baselineskip}}
}

\author{\textbf{Author} \\
    Victor Elvira Fernández, Tomás Ruiz Rojo, Juan Horrillo Crespo \\
    Universidad de Valladolid \\
    \date{\today}}

\maketitle
\newpage

\tableofcontents
\newpage

% ------------------------------------------------------------------------------



\subsubsection*{Ejercicio 1}
$X_1, \dots, X_n$ v.a.i.i.d. $\quad E(X_i)=\mu,\quad Var(X_i)=\sigma^2 \quad \forall i=1 \dots n$
\setlength{\parskip}{1em}

b) $T(X_1,\dots,X_n)=\frac{T(X_1 + \dots + X_{\frac{n}{2}})}{\frac{n}{2}}$\\
$T(X_1,\dots,X_n) \xrightarrow{P} \mu \quad$ El estimador es consistente

c) $T(X_1,\dots,X_n)=X_1$\\
El estimador no es consistente ya que $X_1$ no depende de n. Lo mínimo para que
pueda converger es que dependa de n.

d) $T(X_1,\dots,X_n)=2 \cdot \sum_{i=1}^{n}\frac{i \cdot X_i}{n \cdot (n+1)}$\\
No podemos usar la ley de los grandes números porque $X_1, 2 \cdot X_2, 3 \cdot X_3$ no son v.a.i.i.d. Usaremos la convergencia en media cuadrática.

\[
T_n(x) \xrightarrow{m.c.} \mu 
\left\{
\begin{array}{l}
    E(T_n(X)) \xrightarrow[n \to \infty]{} \mu \\
    Var(T_n(X)) \xrightarrow[n \to \infty]{} 0
\end{array}
\right.
\]
\(
E(T_n(x))=\frac{2}{n \cdot (n+1)} \cdot E((\sum_{i=1}^{n} i) \cdot X_i)
=\frac{2}{n \cdot (n+1)} \cdot (\sum_{i=1}^{n} i) \cdot  E(X_i)=
\frac{2 \cdot \mu}{n \cdot (n+1)} \cdot \sum_{i=1}^{n} i \\
=\frac{2 \cdot \mu}{n \cdot (n+1)} \frac{n \cdot (n+1)}{2}= \mu
\)

Se cumple que la media tiende a mu cuando n tiende a infinito.

\(
Var(T_n(x))=\frac{4}{n^2 \cdot (n+1)^2} \cdot \sum_{i=1}^{n} Var(i\cdot X_i)
= \frac{4}{n^2 \cdot (n+1)^2} \cdot \sigma^2 \cdot \sum_{i=1}^{n} i^2 \\
= \frac{4}{n^2 \cdot (n+1)^2} \cdot \sigma^2 \cdot \frac{n \cdot (n+1) \cdot (2n+1)}{6}
=\frac{2}{3}\cdot \frac{2n+1}{n^2+n} \sigma^2 \xrightarrow[n \to \infty]{} 0
\)

Se cumple que la media tiende a 0 cuando n tiende a infinito.

El estimador es consistente.

\subsubsection*{Ejercicio 3}


% ------------------------------------------------------------------------------
% Reference and Cited Works
% ------------------------------------------------------------------------------
\newpage
\begin{thebibliography}{9}
    \bibitem{diapos1}
    ¿Quién?, \\ \textit{¿Qué?}. \\ ¿Dónde?.
\end{thebibliography}

% ------------------------------------------------------------------------------

\end{document}
