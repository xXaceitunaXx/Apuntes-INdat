
\subsection*{Ejercicio 1}

Sean $X_1, \dots, X_n$ v.a.i.i.d. donde $E(X_i)=\mu,\quad Var(X_i)=\sigma^2 \quad \forall i=1 \dots n$. Determinar si los siguientes estimadores son consistentes
\begin{enumerate}[label=\alph*)]\setcounter{enumi}{1}
    \item $T(X_1,\dots,X_n)=\frac{T(X_1 + \dots + X_{\frac{n}{2}})}{\frac{n}{2}}$\\
          Vemos que
          \[
              T(X_1,\dots,X_{\frac{n}{2}}) = \frac{\sum_{i=1}^{\frac{n}{2}}}{\frac{n}{2}} = \bar{X}_{\frac{n}{2}} \xrightarrow{P} \mu
          \]
          Por tanto el estimador es consistente

    \item $T(X_1,\dots,X_n)=X_1$\\
          El estimador no es consistente ya que $X_1$ no depende de n. Lo mínimo para que pueda converger es que dependa de n.

    \item $T(X_1,\dots,X_n)=2  \sum_{i=1}^{n}\frac{i  X_i}{n(n+1)}$\\
          No podemos usar la ley de los grandes números porque $X_1, 2X_2, 3X_3, \dots$ no son v.a.i.i.d. por tanto usaremos la convergencia en media cuadrática.

          \[
              T_n(x) \xrightarrow{m.c.} \mu
              \left\{
              \begin{array}{l}
                  E(T_n(X)) \xrightarrow[n \to \infty]{} \mu \\
                  Var(T_n(X)) \xrightarrow[n \to \infty]{} 0
              \end{array}
              \right.
          \]
          \[
              E(T_n(x))=\frac{2}{n  (n+1)}  E((\sum_{i=1}^{n} i)  X_i)=\frac{2}{n  (n+1)}  (\sum_{i=1}^{n} i)   E(X_i)=
          \]
          \[
              =\frac{2  \mu}{n  (n+1)}  \sum_{i=1}^{n} i=\frac{2  \mu}{n  (n+1)} \frac{n  (n+1)}{2}= \mu
          \]
          Se cumple que la media tiende a mu cuando n tiende a infinito.

          \[
              Var(T_n(x))=\frac{4}{n^2  (n+1)^2}  \sum_{i=1}^{n} Var(i X_i)= \frac{4}{n^2  (n+1)^2}  \sigma^2  \sum_{i=1}^{n} i^2
          \]
          \[
              =\frac{4}{n^2  (n+1)^2}  \sigma^2  \frac{n  (n+1)  (2n+1)}{6}=\frac{2}{3} \frac{2n+1}{n^2+n} \sigma^2 \xrightarrow[n \to \infty]{} 0
          \]
          Como se cumple que la media tiende a 0 cuando n tiende a infinito el estimador es consistente.
\end{enumerate}

\subsection*{Ejercicio 3}
Sean $X_1, \dots, X_n$ v.a.i.i.d. con distribución uniforme $U(0,\theta)$, ¿$2 \bar{X}$ es consistente para $\theta$?
\[
    E(X_i)=\frac{a+b}{2}=\frac{\theta}{2};\quad Var(X_i)=\frac{(b-a)^2}{12}
\]
Entonces
\[
    \bar{X} \xrightarrow[n \to \infty]{P} \implies 2 \bar{X} \xrightarrow{P} \theta
\]

Por tanto $2 \bar{X}$ es consistente para $\theta$

\newpage

\subsection{Información de Fisher}

La información de Fisher $I_x(\theta)$ es la matriz que mide la cantidad
de información que una m.a.s. contiene sobre el estimador.

\textbf{\textit{Definición: }} Sea $X=(X_1 \dots X_n)$ v.a.i.i.d. con distribución $P_\theta \in P=\{P_\theta : \theta \in \Theta \subseteq \mathbb{R}\}$ con función de densidad $f(X,\theta)$ y en la que existe $\frac{d f(x,\theta)}{d \theta}$, la información de Fisher sobre $\theta$ contenida en X es:
\[
    I_x(\theta) = Var_\theta (S(\theta,X))
\]
\[
    Score=S_x(\theta)=S(\theta,X)=\frac{d}{\mathrm{d\theta}}\log f(x,\theta)
\]
\subsection{Condiciones de regularidad de Craner-Rao (CRCR)}

Llamaremos familias regulares a aquellas familias en las que se verifican las
condiciones de regularidad de Craner-Rao.
\\ Estas son las familias con las que trabajaremos.
\\ \\\textbf{Condiciones de regularidad de Craner-Rao:}
\begin{enumerate}
    \item El espacio paramétrico es un intervalo de $\mathbb{R}$.
    \item El soporte de la distribución no depende del parámetro $\theta$.
          Por ejemplo $x=\{x:f(x,\theta) > 0 \}$, no depende de $\theta$ y sería regular. En cambio U(0,$\theta$)
          tiene un soporte que depende de $\theta$, por lo que no es regular
    \item Se pueden calcular las dos primeras derivadas bajo el signo integral.
          Además se puede intercambiar la derivada con el signo integral.
          \[
              \frac{d}{d \theta} \int_{x} f(x,\theta)  \,dx = \int_{x} \frac{d}{d \theta} f(x,\theta)  \,dx
          \]
    \item $T_n(x)$ es un estimador insesgado para $\theta$ o $g(\theta)$.
\end{enumerate}

Bajo las condiciones de regularidad de Craner-Rao, podemos definir la cantidad
de información esperada como:
\[
    I_x(\theta)=E_\theta(S(\theta,X)^2)=E_\theta\left(\left(\frac{d}{d \theta} \log f(x,\theta)\right)^2\right)
\]

\begin{proofs}
    Deberemos probar que $E_\theta(\left(\frac{d}{d \theta} \log f(x,\theta)\right))=0$. Si lo conseguimos entonces, $Var(S(\theta,x))=E(S(\theta,x)^2)-E(S(\theta,x))^2=E(S(\theta,x)^2)$
    \[
        E_\theta\left(\left(\frac{d}{d \theta} \log f(x,\theta)\right)\right)=\int_{x} E_\theta\left(\left(\frac{d}{d \theta} \log f(x,\theta)\right)\right)  f(x,\theta) \,dx =
    \]
    \[
        =\int_{x} \frac{\frac{d}{d \theta} f(x,\theta)}{f(x,\theta)}  f(x,\theta) \,dx=\int_{x} \frac{d}{d \theta} f(x,\theta) \,dx = 0
    \]
\end{proofs}

Si se verifican las condiciones de regularidad de Crner-Rao, otra forma alternativa de calcular la información de Fisher es:
\[
    I_x(\theta)=-E\left(\frac{d^2}{d \theta} \log f(x,\theta)\right)
\]

\begin{proofs}
    Breve paso previo:
    \[
        \frac{d}{d \theta}\left(\frac{d}{d \theta} \log f(x,\theta)\right)=\frac{d}{d \theta}\left(\frac{\frac{d}{d \theta} f(x,\theta)}{f(x,\theta)}\right)=
    \]
    donde derivando el cociente
    \[
        =\frac{\frac{d^2}{d \theta} f(x,\theta) f(x,\theta) -\left(\frac{d}{d \theta} f(x,\theta)\right)^2}{f(x,\theta)^2}=\frac{\frac{d^2}{d \theta} f(x,\theta)}{f(x,\theta)} -\left(\frac{\frac{d}{d \theta} f(x,\theta)}{f(x,\theta)}\right)^2
    \]
    Demostración:
    \[
        E(\frac{d^2}{d \theta} \log f(x,\theta))=\int_{x} \frac{d}{d \theta}\left(\frac{d}{d \theta} \log f(x,\theta)\right)  f(x,\theta) \,dx =
    \]
    \[
        =\int_{x} \frac{d^2}{d \theta} f(x, \theta) \,dx - \int_{x}\left(\frac{\frac{d}{d \theta} f(x,\theta)}{f(x,\theta)}\right)^2  f(x,\theta) \,dx
    \]
    \[
        = -\int_{x}(\frac{\frac{d}{d \theta} f(x,\theta)}{f(x,\theta)})^2  f(x,\theta) \,dx=-I_x(\theta)
    \]
\end{proofs}

\textbf{\textit{Propiedades:}}

Sean X e Y dos variables independientes de la misma familia de distribuciones

$X \sim P_\theta, \quad \theta \in \Theta \subseteq \mathbb{R}, \quad f(x,\theta), I_x(\theta)$

$Y \sim Q_\theta, \quad \theta \in \Theta \subseteq \mathbb{R}, \quad g(y,\theta), I_y(\theta)$

Entonces
\begin{enumerate}
    \item Propiedad de la información de Fisher conjunta: $I_{xy}(\theta) = I_x(\theta)+I_y(\theta)$
          \begin{proofs}
              \[
                  f_{xy}(x,y,\theta)=f_x(x,\theta)  f_y(y,\theta)
              \]
              \[
                  I_{xy}(\theta)=Var(\frac{d}{d \theta} \log (f_x(x,\theta)  f_y(y,\theta)))=Var(\frac{d}{d \theta} (\log f_x(x,\theta) + \log f_y(y,\theta)))
              \]
              Y por las propiedades de la varianza: $Var(X+Y)= Var(X)+Var(Y)$ si $X$ e $Y$ son independientes
              \[
                  =Var(\frac{d}{d \theta} (\log f_x(x,\theta))) + Var(\frac{d}{d \theta} (\log f_y(y,\theta)))=I_x(\theta)+I_y(\theta)
              \]

          \end{proofs}



    \item Sean $X_1, \dots, X_n$ m.a.s. v.a.i.i.d. $P_\theta,\theta \in \Theta$ con $f(x,\theta)$ de una familia regular:
          \[
              I_{X_1,\dots,X_n}=n  I_{X_1}(\theta)
          \]
\end{enumerate}

\subsection*{Ejemplo con la distribucion de Bernoulli:}

\(
X \sim B(p) \to f(x,p)=p^x (1-p)^{1-x}  \quad x=0,1
\\ \\ \text{Usando la primera definición de }I_x(p):
\\ \\ I_x(p)=Var((S_x(p)))=Var(\frac{d}{dp} \log f(x,p))
\\ log f(x,p)=x \log p + (1-x)\log(1-p)
\\ S_x(p)=\frac{d}{dp}(x \log (p)+ (1-x)\log (1-p))
\\ Var(S_x(p))=Var(\frac{x-p}{p(1-p)})=\frac{Var(x)}{p^2(1-p)^2}=
\frac{p(1-p)}{p^2(1-p)^2}=\frac{1}{p(1-p)}
\)

\subsection*{Ejemplo con la distribucion de Poisson:}

\(
I_x(\lambda)=Var(S_x(\lambda))
\\ f(x,\lambda) = \frac{\lambda^x  e^{-\lambda}}{x!}
\\ S_x(\lambda) = \frac{d}{d \lambda} \log\left(\frac{\lambda^x  e^{-\lambda}}{x!}\right) = \frac{x-\lambda}{\lambda}
\\ I_x(\lambda) = Var(\frac{x-\lambda}{\lambda})=\frac{1}{\lambda^2}Var(x-\lambda)
=\frac{\lambda}{\lambda^2}=\frac{1}{\lambda}
\)

\subsection*{Ejemplo con n muestras de la distribucion de Bernoulli:}

\(
X_1,\dots,X_n \quad v.a.i.i.d. \quad B(p)
\\ \\ f(X_1,\dots,X_n)=\prod_{i=1}^{n} p^{x_i}(1-p)^{1-x_i}=p^{\sum_{i=1}^{n} x_i}
(1-p)^{n-\sum_{i=1}^{n} x_i}
\\ \log f(x_1,...,x_n)=(\sum_{i=1}^{n}x_i) \log p + (n-\sum_{i=1}^{n}x_i) \log (1-p)
\\ S_{x_1,...,x_n}(p)=\frac{d}{dp} \log f(X_1,...,X_n)=\frac{\sum_{i=1}^{n}x_i - np}{p(1-p)}
\\ Var(\frac{\sum_{i=1}^{n}x_i - np}{p(1-p)})=\frac{\sum_{i=1}^{n}(Var x_i)}{p^2(1-p)^2}=\frac{n}{p(1-p)}
I_{x_1,...,x_n}(p)=n  I_{x_1}(p)=\frac{n}{p(1-p)}
\)

\subsection*{Ejemplo con la exponencial:}

\(
X \sim exp(\lambda)
\\ \\ f(x,\theta)=\lambda  e^{-\lambda x}
\\ \log f(x,\theta)=\log \lambda - \lambda X
\\S_x(\lambda)=\frac{1}{\lambda}-X
\\ I_x (\lambda)=-E_\lambda(\frac{d}{d \lambda} S_x(\lambda))=-E_\lambda(\frac{-1}{\lambda^2})
=\frac{1}{\lambda^2}
\)

