\setlength{\parskip}{1em}

¿Cómo hacemos inferencia para una función del parámetro?
Utilizando el delta método s-variante.

Partimos de la situacion $H_0:g(\theta)=0$.

Si g es una función de $\theta$ tal que g:$\theta \to \mathbb{R}^p \quad p \leq S, \quad g(\theta)=(g_1(\theta),\dots,g_p(\theta))$
y sea $G(\theta)$ la matriz p x s de las primeras derivadas respecto a $\theta$:

\[
G(\theta)=
\begin{pmatrix}
    \frac{d}{d \theta_1} g_1(\theta) & \dots & \frac{d}{d \theta_s} g_1(\theta)\\
    \dots &\dots & \dots \\
    \frac{d}{d \theta_1} g_p(\theta) & \dots &\frac{d}{d \theta_s} g_p(\theta)
\end{pmatrix}
\]

Si $\hat{\theta}$ tiene distribución asintótica ($N_s(\theta_s,V)$)
y $g(\hat{\theta})$ también tiene distribución asintótica ($N_p(g(\theta_0)),G_{pxs}(\theta_0)\cdot V \cdot G(\theta)'$)

\[
W=(g(\hat{\theta}-g(\theta_0)))'(G(\hat{\theta})\cdot \hat{V} G(\hat{\theta})')^-{-1}(g(\hat{\theta}-g(\theta_0))) \sim \chi^2_p
\]

\subsubsection*{Ejercicio 1}
Se analizan 100 lotes con 10 muestras cada uno $\sum_{i=1}^{100}Y_i=12 \quad Y_1,\dots,Y_{100} \sim B(\theta)$.
\\$\theta$="probabilidad de que la toxina esté presente en el lote"
\\ p="probabilidad de que la toxina esté presente en una muestra individual"

\textbf{Apartado a}.

Nos piden hacer inferencia sobre p, sabiendo que p es función de $\theta$.

\(
P(Y_1=0)=1-\theta=(1-p)^{10} \quad P(Y_i=1)=\theta
\)

Ya que ya tenemos la relación, hacemos inferencia sobre p.

\(
L(\theta,Y_1,\dots,Y_{100})=\prod_{i=1}^{100} \theta^{Y_i}(1-\theta)^{1-Y_i}=\theta^{\sum_{i=1}^{100} Y_i}(1-\theta)^{100-\sum_{i=1}^{100} Y_i}
\\ \log L(\theta,Y_1,\dots,Y_{100})=(\sum_{i=1}^{100}Y_i) \log \theta + (100-\sum_{i=1}^{100}Y_i) \log (1-\theta)
\\ \frac{d}{d \theta} log L(\theta,Y_1,\dots,Y_{100})= \frac{\sum_{i=1}^{100}Y_i}{\theta} - \frac{100 - \sum_{i=1}^{100}Y_i}{1-\theta}=0
\)

El EMV es invariante por transformación: $EMV \to \hat{\theta}=\bar{Y}$

p=g($\theta$)=1-$(1-\theta)^{0.1}$ $\Longrightarrow \hat{p}=g(\hat{\theta})=1-(1-\bar{Y})^{0.1}$

\textbf{Apartado b}.

Sabemos que $\hat{\theta}=\bar{Y}$ y que tiene distribución asintóticamente normal.

\[
I_n(\theta)=\frac{n}{\theta(1-\theta)} \quad \hat{\theta}\simeq N(\theta,\frac{1}{I_n(\theta)}) = N(\theta,\frac{\theta(1-\theta)}{n})
\]

Sabiendo esto, ¿cuál será la distribución de p?

\[
\hat{p}=g(\hat{\theta}) \simeq N(g(\theta),\frac{\hat{\theta}(1-\hat{\theta})}{n}\cdot (g'(\theta))^2)
\]

¿Y cual sería un intervalo de confianza de Wald con confianza del 95$\%$ para p?

\[
\hat{p} \pm qnorm(0.975)\sqrt{Var(\hat{p})}
\]

\textbf{Apartado c}.
Calcular el ICRV con confianza 0.95 para p.

$Q_L \approx Q_W \sim \chi^2_1$

Usamos la fórmula que sabemos para calcular este intervalo para $\theta$.

\[
ICRV=\{ \theta_0:2[\log L(\hat{\theta},X) - log L(\theta_0,X)] \leq qchisq(0.95,1)\}
=\{ \theta_0:\log L(\theta_0,X) \geq \log L(\bar{Y},X)-\frac{qchisq(0.95,1)}{2}\}
\]

Como los intervalos de confianza basados en RV son invariantes por transformación, el intervalo de confianza RV para p es:

\[
[1-(1-L)^{0.1},1-(1-M)^{0.1}]
\]

(L y M son los puntos entre los que se cumple que $log L(\theta_0,X) \geq \log L(\bar{Y},X)-\frac{qchisq(0.95,1)}{2}$)