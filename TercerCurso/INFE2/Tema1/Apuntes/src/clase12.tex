
\section{Maximización de la verosimilitud}

En la mayoría de casos el EMV se obtiene de forma explícita. Sin embargo, en muchos casos prácticos la ecuación (o ecuaciones) de verosimilitud son muy complejas.
Por ejemplo, para obtener el EMV en R podemos utilizar la función \textit{optim}. En teoría veremos dos algoritmos distintos: el de \textbf{Newton-Raphson} y el \textbf{algoritmo EM}. 

\subsection*{Algoritmo de Newton-Raphson (NR)}

Está basado en la aproximación analítica de la función objetivo vía la aproximación lineal de su derivada. \\

Sean $X_1, X_2,...,X_n$ v.a.i.i.d, con $P_\theta\ \theta\in\Theta\subseteq \mathbb{R}^s$ y $\theta=(\theta_1,...,\theta_s)$ donde la función objetivo es $l(\theta,x)=ln(L(\theta,x))$.
El algoritmo NR busca el máximo de la log-verosimilitud a través de la aproximación de su derivada: 
$$\frac{\partial}{\partial\theta}ln(L(\theta,x))=\Big(\frac{\partial}{\partial\theta_1}ln(L(\theta,x)),...,\frac{\partial}{\partial\theta_s}ln(L(\theta,x))\Big)$$
basada en el desarrollo de Taylor.\\

Partimos de un valor inicial "razonable" del parámetro: $\theta_0$ y aproximamos la función por el desarrollo de Taylor.

$$\frac{\partial}{\partial\theta}ln(L(\theta,x))\approx\frac{\partial}{\partial\theta}ln(L(\theta_0,x))+H(\theta_0,x)(\theta-\theta_0)$$
Siendo $H(\theta_0,x)=\Big{\{}\frac{d^2}{d\theta_i d\theta_j}ln(L(\theta,x))\Big{\}}_{i,j}$ la matriz Hessiana de la función de log-verosimilitud. \\

Este resultado es váido para todo $\theta$ y se deduce:

$$\frac{\partial}{\partial\theta}ln(L(\hat\theta,x))\approx\frac{\partial}{\partial\theta}ln(L(\theta_0,x))+H(\theta_0,x)(\hat\theta-\theta_0)$$

Despejando obtenemos lo siguiente:

$$\hat\theta=\theta_0-H(\theta_0,x)^{-1}\frac{\partial}{\partial\theta}ln(L(\theta_0,x))=\theta^{(1)}$$

Si repetimos k veces el paso anterior llegaremos a $\theta^{(k)}$, por lo que la fórmula de actualización del algoritmo en la siguiente iteración será:

$$\theta^{(k+1)}=\theta^{(k)}-H(\theta^{(k)},x)^{-1}\frac{\partial}{\partial\theta}ln(L(\theta^{(k)},x))$$

Es importante la elección del punto inicial para evitar convergencia a \textbf{óptimos locales}. \pagebreak

\subsection*{Algoritmo EM (\textit{Expectation-maximization})}

Utilizado en situaciones donde los datos que tenemos se pueden considerar \textbf{incompletos} (censurados). 
Es decir, cuando el experimento tiene dos variables aleatorias $Y$ y $B$ pero solamente se observa $Y=y$. $B$ no se observa y los datos completos serán $X=(Y,B)$. 

Sean $X_1, X_2,...,X_n$ v.a.i.i.d, con $P_\theta\ \theta\in\Theta\subseteq \mathbb{R}^s$ con $X_i=(Y_i,B_i)$ y función de densidad $f(x,\theta)=f(y,b,\theta)$ donde solo $Y_1,...,Y_n$ se observan con función de densidad $g(Y,\theta)$.
En este contexto:
\begin{itemize}
    \item f es la función de densidad de los datos \textbf{completos}.
    \item g es la función de densidad de los datos \textbf{observados}.
\end{itemize}
Ambas densidades dependen de $\theta$ pero la densidad de $X$ depende también de $B$, datos que no hemos observado. En estos casos, el EMV lo obtendremos \textbf{siempre} de los \textbf{datos observados}.
\\

La función de verosimilitud de los datos observados es:
$$L(Y_1,...,Y_n,\theta)\propto \prod_{i=1}^{n}g(Y_i,\theta)$$ $$ln(L(Y,\theta))=\sum_{i=1}^{n}ln(g(Y_i,\theta))$$ Y el EMV será (como siempre): $\hat\theta=\underset{\theta}{arg\ max}[ln(L(Y,\theta))]$
\\

Sin embargo, hay situaciones en las que es difícil resolver y plantear las ecuaciones de verosimilitud. El algoritmo EM nos permite calcular la función de verosimilitud a partir de los datos completos cuando es complicado hacerlo a partir de los observados.
$$L(X_1,...,X_n,\theta)=L((Y_1,B_1),...,(Y_n,B_n),\theta)=\prod_{i=i}^{n}f(Y_i,B_i,\theta)$$

El algoritmo EM obtiene un valor aproximado del EMV que es el máximo de los datos observados a partir del EMV de los datos completos.

\subsubsection*{Desarrollo del algoritmo EM}
Se puede calcular la esperanza de $ln(L(X,\theta))$ condicionada a $Y=y$ dado un valor de $\theta$. A partir de un \textbf{valor inicial} $\theta_0$, el algoritmo establece dos pasos:
\begin{enumerate}
    \item \textbf{Esperanza (expectation):} Calcular $E_{\theta_0}\Big(ln(L(X,\theta))\big/Y\Big)$ y sustituir la probabilidad de los valores no observados por el valor esperado condicionado a lo observado.
    \item \textbf{Maximización (maximization):} Obtener $\theta^{(n)}$ en el punto en el que se alcanza el máximo en la log-verosimilitud habiendo sustituido la expresión anterior.
\end{enumerate} 
Este proceso se repite \textbf{iterativamente} $k$ veces. 

\pagebreak

\hspace{-1cm}\noindent\begin{tabular}{r}
    \textbf{Ejemplo}  \\ \hline \ \\
\end{tabular}

\textit{Este ejemplo es el de la mixtura de la práctica 3 del laboratorio}.\\
Sean $f_1(Y)=P_p(Y=y)=\begin{cases}
    1 & \text{si } y=0 \\
    0 & \text{si } y \neq 0
\end{cases}$ y $f_2(Y)=P_\lambda(Y=y)=e^{-\lambda}\frac{\lambda^y}{y!}$, $y=0,1,...$

\ \\
Primero obtenemos las fórmulas de adecuación del algoritmo para $p$ y $\lambda$.
\begin{itemize}
    \item Datos observados: $Y_i$
    \item Datos completos: $X_i=(Y_i,B_i)$
    \item $B_i$ son datos NO observados
\end{itemize}
Función de densidad conjunta: $ f(x,p,\lambda)=f(y,b,p,\lambda)=\begin{cases}
    p & \text{si } b_i=1\ y\ y_i=0 \\
    (1-p)e^{-\lambda}\frac{\lambda^y}{y!} & \text{si } b_i=0\ y\ y_i \neq 0
\end{cases} $

$$L(p,\lambda,b,y)\propto\prod_{i=1}^{n}p^{b_i}\Big[(1-p)e^{-\lambda}\frac{\lambda^y}{y!}\Big]^{1-b_i}=p^{\sum b_i}(1-p)^{n-\sum b_i}e^{-\lambda(n-\sum b_i)}\lambda^{\sum y_i(1-b_i)}$$
$$ln(L(p,\lambda,b,y))=ln(p)\sum b_i+(n-\sum b_i)ln(1-p)-\lambda(n-\sum b_i)+ln(\lambda)\sum y_i(1-b_i)$$\\\ \\
\underline{Paso 1, Esperanza:} Dados $p_0$ y $\lambda_0$ sustituir $b_i$ por $E_{p_0,\lambda_0}\Big(B_i\Big/Y_i=y\Big)$:
$$E_{p_0,\lambda_0}\Big(ln(L(p_0,\lambda_0,b,y))\Big/Y_1,...,Y_n\Big)=ln(p)\Bigg[\sum E_{p_0,\lambda_0}\Big(B_i\Big/Y_i=y\Big)\Bigg]+ln(1-p)\Bigg[n-\sum E_{p_0,\lambda_0}\Big(B_i\Big/Y_i=y\Big)\Bigg]$$








