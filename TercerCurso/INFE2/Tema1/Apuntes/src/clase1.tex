\section{Propiedades asintóticas del EMV y del estadístico RV xd}


\subsection{Estimadores en muestras grandes}

Nuestro principal objetivo es analizar cómo se comporta la muestra cuando su tamaño es tan grande como nosotros queramos.
\setlength{\parskip}{1em}   % Aumentar espacio entre párrafos

Situación:
\setlength{\parskip}{0em}   % Aumentar espacio entre párrafos

$ X_1, X_2, \dots, X_n $ variables aleatorias independientes igualmente distribuidas (v.a.i.i.d.) tal que $ P_\theta:\theta  \in  \Theta$.

Tenemos nuestra muestra aleatoria simple (m.a.s) con $ n $ grande y nos interesa estimar $ \theta $ (o $ g(\theta) $).

\setlength{\parskip}{1em}   % Aumentar espacio entre párrafos
Cuando el tamaño de la muestra aumenta, también aumenta la información disponible de $ \theta $ (o $ g(\theta) $), por lo tanto se espera que la estimación sea más precisa.

Si tenemos un estimador $ T(X_1, \dots, X_n) $ razonable, cuando $ n $ aumenta, el estimador $ T(X_1, \dots, X_n) $ deberá ser más preciso.

\setlength{\parskip}{0em}   % Aumentar espacio entre párrafos
Lo que podemos esperar es que $ T(X_1, \dots, X_n) $ esté próximo a $ \theta $ con mayor probabilidad.

\subsubsection{Consistencia de un estimador}

Se dice que un estimador es consistente si cumple:
\setlength{\parskip}{1em}

\[\forall \epsilon > 0, \; \delta > 0 \quad \exists N: n \geq N\]
\[P_\theta\left(|T(X_1, \dots, X_n) - \theta| < \epsilon\right) \geq 1 - \delta\]

\textbf{\textit{Definición: }} $T_n(X)$ es un estimador consistente para $\theta$ (o $g(\theta)$) si:

\[
    \forall \epsilon > 0 \quad P_\theta\left(|T(X_1, \dots, X_n) - \theta| \leq \epsilon\right) \xrightarrow{n \to \infty} 1
\]

es decir, si:

\[
    T_n(x) \xrightarrow{\underset{n \to \infty}{\text{P}}} \theta \quad \text{o} \quad \lim_{n \to \infty} \forall \epsilon > 0 \quad P_\theta\left(|T(X_1, \dots, X_n) - \theta| \leq \epsilon\right) = 1
\]

\newpage
\subsubsection*{Estrategias para comprobar si un estimador es consistente}

\begin{enumerate}
    \setlength{\parskip}{1em}
    \item \textbf{Si converge en probabilidad.}
          Resultado:
          \[
              Si \quad T_n(x) \xrightarrow{\underset{n \to \infty}{\text{P}}} \theta \quad \text{y} \quad T_n(x) \xrightarrow{\underset{n \to \infty}{\text{P}}} \theta'
          \]
          \begin{itemize}
              \item $T_n(x)+G_n(x) \xrightarrow{\underset{n \to \infty}{\text{P}}} \theta + \theta '$
              \item $T_n(x)G_n(x) \xrightarrow{\underset{n \to \infty}{\text{P}}} \theta\theta '$
              \item Si $\theta' \neq 0, \frac{T_n(x)}{G_n(x)} \xrightarrow{\underset{n \to \infty}{\text{P}}} \frac{\theta}{\theta '}$
          \end{itemize}

    \item \textbf{Utilizando la ley débil de los grandes números.}
          $X_1, X_2, \dots, X_n$ i.i.d. con $E(X_i)=\mu < \infty$
          \setlength{\parskip}{0em}
          \[\frac{1}{n}\sum_{i=1}^{n} X_i \xrightarrow{\underset{n \to \infty}{\text{P}}} \mu\]

          \setlength{\parskip}{1em}
          Los momentos muestrales son estimaciones consistentes de los correspondientes momentos poblacionales

    \item \textbf{Utilizando la convergencia en media cuadrática.}
          Un estimador $T(X_1, \dots, X_n)\xrightarrow{\underset{n \to \infty}{\text{m.c.}}} \theta$ si:
          \begin{itemize}
              \item $E(T(X_1, \dots, X_n)) \xrightarrow{{n \to \infty}} \theta$, sesgo($T(X_1, \dots, X_n)$) $\xrightarrow{{n \to \infty}} 0$
              \item $\text{Var}(T(X_1, \dots, X_n)) \xrightarrow{{n \to \infty}} 0$
          \end{itemize}
          Es decir que si ECM ($T(X_1, \dots, X_n)$) $\xrightarrow{{n \to \infty}} 0$, el estimador converge en media cuadrática. Un estimador insesgado sería consistente si su varianza converge a 0 con $n \to \infty$

          El resultado es que la convergencia en media cuadrática es más fuerte que la convergencia en probabilidad.
\end{enumerate}
\noindent\rule{\textwidth}{0.2pt} % Línea de separación

\subsubsection*{Ejercicio 1}
$X_1, \dots, X_n $ v.a.i.i.d \quad E(X_i)=\mu, \text{Var}(X_i)=\sigma^2 \quad \forall i=1, \dots, n

¿Es consistente $T_n(x) = \frac{X_1 + X_2 + \dots + X_n}{\frac{n}{2}}$?

Usando la ley de los grandes números:
$\frac{1}{n}\sum X_i \xrightarrow{\text{P}} \mu \quad T_n(x)\xrightarrow{\text{P}}2\mu$

Resultado: El estimador $T_n(x)$ NO es consistente

\newpage

La consistencia por si sola no es tan interesante, ya que si $T_n \text{es consistente para } \theta$, nos dice que para n grande los errores serán pequeños pero no nos permite conocer el orden del error $\left(\frac{1}{n}, \frac{1}{\sqrt{n}},\frac{1}{\log(n)}, \dots\right)$

Si $\{K_n\}_{n=1}^{\infty}$ es una sucesión de reales positivos y $\epsilon>0$ definimos $P_n(\epsilon)$.

$P_n(\epsilon)=P_\theta(|T_n(x)-\theta| \leq \frac{\epsilon}{K_n})$

Habiendo definido $P_n(\epsilon)$, ¿qué pasará con  $P_n(\epsilon)$ cuando n sea grande?
\begin{enumerate}
    \item Si $K_n$ crece "lentamente" (por ejemplo:  $K_n=\log(n)$), el error disminuye "lentamente" a medida que aumenta n $P_n(\epsilon) \xrightarrow{{\text{n} \to \infty}} 1$. Si $K_n$ crece lentamente, $ \frac{\epsilon}{K_n}$ es más grande, lo que facilitará que el error esté por debajo del umbral.
    \item Si $K_n$ crece "rápido" (por ejemplo:  $K_n=n$), el error disminuye más rápido a medida que aumenta n $P_n(\epsilon) \xrightarrow{{\text{n} \to \infty}} 0$.  Si $K_n$ crece rápido, $ \frac{\epsilon}{K_n}$ se hace muy pequeño y se hace muy difícil que el error sea tan pequeño.
    \item Casos intermedios.

          Si $K_n$ crece "adecuadamente", $P_n(\epsilon) \xrightarrow{{\text{n} \to \infty}} H(\epsilon) \in (0,1)$. Decimos que el error converge a 0 a velocidad $\frac{1}{K_n}$
\end{enumerate}
De manera resumida:

\[
    P_n(\epsilon) = P_\theta(K_n|T_n(x) - \theta| \leq \epsilon) \xrightarrow{{n \to \infty}

    \left\{
    \begin{array}{l}
        0 \text{ si } K_n \xrightarrow{\infty} \text{ "rápido"}                                  \\[1em]
        0 \leq P_n(\epsilon) \leq 1 \text{ si } K_n \xrightarrow{\infty} \text{ "adecuadamente"} \\[1em]
        1 \text{ si } K_n \xrightarrow{\infty} \text{ "lento"}
    \end{array}
    \right\}
\]

La idea es que al multiplicar $K_n$, se amplifica la velocidad de convergencia de los errores a 0. Si elegimos $K_n$ adecuadamente, de forma que $P_n(\epsilon)$ sea menor que 1, podemos controlar la velocidad a la que los errores tienden a 0, mejorando la precisión.
