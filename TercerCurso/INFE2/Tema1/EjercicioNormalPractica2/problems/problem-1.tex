Formalizamos el contraste de hipotesis:

\[
    H_0\text{:}\quad \frac{\mu}{\sigma} = 1
\]

\[
    H_1\text{:}\quad \frac{\mu}{\sigma} \neq 1
\]

Estamos ante un caso de inferencia multiparamétrica pues tenemos una hipótesis compuesta. Usaremos el test de Wald para realizar el contraste.

Para ello definiremos una función $g$ que transforme nuestro espacio paramétrico a $\mathbb{R}$ con el objetivo de calcular
el estadístico de Wald.


Como la hipotesis a realizar es que $\frac{\mu}{\sigma} = 1$, podemos reescribirla como $\mu = \sigma$, y de forma inmediata $\mu - \sigma = 0$.

Por ello definimos $g$ de la siguente manera:

\[
    g: \theta \longrightarrow \mathbb{R} \text{ donde } g(\theta) = \mu - \sigma
\]

Reescribimos la hipótesis para nuestra función $g$:

\[
    H_0\text{:}\quad g(\theta) = 0
\]

\newpage

Hallamos la distribución de la función del parámetro utilizando el $\delta$-método

\[
    \delta
\]