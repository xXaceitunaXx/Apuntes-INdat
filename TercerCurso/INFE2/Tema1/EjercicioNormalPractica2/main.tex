\documentclass[letterpaper, 11pt]{extarticle}
% \usepackage{fontspec}

% ==================================================

% document parameters
% \usepackage[spanish, mexico, es-lcroman]{babel}
\usepackage[spanish, es-lcroman]{babel}
\usepackage[margin = 1in]{geometry}

% ==================================================

% Packages for math
\usepackage{mathrsfs}
\usepackage{amsfonts}
\usepackage{amsmath}
\usepackage{amsthm}
\usepackage{amssymb}
\usepackage{physics}
\usepackage{dsfont}
\usepackage{esint}

% ==================================================

% Packages for writing
\usepackage{enumerate}
\usepackage[shortlabels]{enumitem}
\usepackage{framed}
\usepackage{csquotes}

% ==================================================

% Miscellaneous packages
\usepackage{float}
\usepackage{tabularx}
\usepackage{xcolor}
\usepackage{multicol}
\usepackage{subcaption}
\usepackage{caption}
\captionsetup{format = hang, margin = 10pt, font = small, labelfont = bf}

% ==================================================

% R code
\usepackage[svgnames]{xcolor}
\usepackage{listings}
\lstset{language=R,
  basicstyle=\small\ttfamily,
  stringstyle=\color{DarkGreen},
  otherkeywords={0,1,2,3,4,5,6,7,8,9},
  morekeywords={TRUE,FALSE},
  deletekeywords={data,frame,length,as,character},
  keywordstyle=\color{blue},
  commentstyle=\color{DarkGreen},
}

% ==================================================

% Citation
\usepackage[round, authoryear]{natbib}

% ==================================================

% Hyperlinks setup
\usepackage{hyperref}
\definecolor{links}{rgb}{0.36,0.54,0.66}
\hypersetup{
  colorlinks = true,
  linkcolor = black,
  urlcolor = blue,
  citecolor = blue,
  filecolor = blue,
  pdfauthor = {Juan},
  pdftitle = {Entrega Practica 2},
  pdfsubject = {INFE2},
  pdfproducer = {LaTeX},
  pdfcreator = {pdfLaTeX},
}
\input{format}
\input{commands}

\begin{document}

\begin{Large}
    \textsf{\textbf{Ejercicio Normal}}

    Entrega práctica 2
\end{Large}

\vspace{1ex}

\textsf{\textbf{Alumno:}} \text{Juan Horrillo Crespo}, \href{mailto:juan.horrillo22@estudiantes.uva.es}{\texttt{juan.horrillo22@estudiantes.uva.es}}

\vspace{1ex}

Sean $X_1, X_2, \dots, X_n$ variables aleatorias idependientes igualmente distribuidas donde $X_i \thicksim N(\mu, \sigma)$ y consideramos $\theta = (\mu, \sigma)$.

Lemas:
\begin{enumerate}
    \item El EMV de la distribución normal es $\hat{\theta} = (\overline{X}, S)$
    \item $\begin{pmatrix}\overline{X}\\S\end{pmatrix} \thicksim N\begin{pmatrix}S^2/n & 0\\0 & S^2/n\end{pmatrix}$
    \item Sea $\hat{p}$ EMV de $p$. Sea $g(\hat{p})$ una función del parámetro. \\ Entonces $g(\hat{p}) \thicksim N(g(p), G(\hat{p})\widehat{Var(p)}G(\hat{p})')$
\end{enumerate}

\begin{problem}{}{test_hipotesis}
Contrastar $H_0\text{:}\quad \frac{\mu}{\sigma} = 1$
\end{problem}

Formalizamos el contraste de hipotesis:

\[
    H_0\text{:}\quad \frac{\mu}{\sigma} = 1
\]

\[
    H_1\text{:}\quad \frac{\mu}{\sigma} \neq 1
\]

Estamos ante un caso de inferencia multiparamétrica pues tenemos una hipótesis compuesta. Usaremos el test de Wald para realizar el contraste.

Para ello definiremos una función $g$ que transforme nuestro espacio paramétrico a $\mathbb{R}$ con el objetivo de calcular
el estadístico de Wald.


Como la hipotesis a realizar es que $\frac{\mu}{\sigma} = 1$, podemos reescribirla como $\mu = \sigma$, y de forma inmediata $\mu - \sigma = 0$.

Por ello definimos $g$ de la siguente manera:

\[
    g: \theta \longrightarrow \mathbb{R} \text{ donde } g(\theta) = \mu - \sigma
\]

Reescribimos la hipótesis para nuestra función $g$:

\[
    H_0\text{:}\quad g(\theta) = 0
\]

\newpage

Hallamos la distribución de la función del parámetro utilizando el $\Delta$-método

\[
    G(\theta) = \begin{pmatrix}\frac{\partial g(\theta)}{\partial \mu}  & \frac{\partial g(\theta)}{\partial \sigma}\end{pmatrix} = \begin{pmatrix}1 & -1\end{pmatrix}
\]

\[
    g(\theta) \thicksim N(g(\theta_0),G(\theta)\widehat{Var(\theta)}G(\theta)') = N\left(0, \begin{pmatrix}1 & -1\end{pmatrix} \frac{S^2}{n} \begin{pmatrix}1 \\ -1\end{pmatrix}\right) =
\]

\[
    = N(0, \frac{3S^2}{2n})
\]

Finalmente calculamos el estadístico de Wald y calculamos el p-valor:

\[
    W_{\text{obs}} = (\overline{X} - S)^2 \frac{2n}{3S^2} \overset{H_0}{\thicksim} \chi^2_1
\]

\vspace{2ex}

\Large{Implementamos la solución en R}

\begin{lstlisting}
# Definimos una funcion que nos calcula el p-valor para un n, una 
# media muestral y una varianza muestral

norm2_pvalue <- function(n, Xs, S) {
    var_emv <- S^2 / n
    G <- matrix(c(1, -1), 1, 2)
    est_var_gtheta <- G %*% var_emv %*% t(G)

    W_obs <- (Xs - S)^2 * 2 * n / (3 * S^2)

    return(1 - pchisq(Q_Wobs, 1))
}
\end{lstlisting}

\end{document}