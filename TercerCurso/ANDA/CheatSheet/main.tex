\documentclass{article}

\usepackage{amsmath, amsthm, amssymb, amsfonts}
\usepackage{thmtools}
\usepackage{graphicx}
\usepackage{setspace}
\usepackage{geometry}
\usepackage{float}
\usepackage{hyperref}
\usepackage[utf8]{inputenc}
\usepackage[english]{babel}
\usepackage{framed}
\usepackage[dvipsnames]{xcolor}
\usepackage{tcolorbox}

\colorlet{LightGray}{White!90!Periwinkle}
\colorlet{LightOrange}{Orange!15}
\colorlet{LightGreen}{Green!15}

\newcommand{\HRule}[1]{\rule{\linewidth}{#1}}

\declaretheoremstyle[name=Theorem,]{thmsty}
\declaretheorem[style=thmsty,numberwithin=section]{theorem}
\tcolorboxenvironment{theorem}{colback=LightGray}

\declaretheoremstyle[name=Proposition,]{prosty}
\declaretheorem[style=prosty,numberlike=theorem]{proposition}
\tcolorboxenvironment{proposition}{colback=LightOrange}

\declaretheoremstyle[name=Principle,]{prcpsty}
\declaretheorem[style=prcpsty,numberlike=theorem]{principle}
\tcolorboxenvironment{principle}{colback=LightGreen}

\setstretch{1.2}
\geometry{
    textheight=9in,
    textwidth=5.5in,
    top=1in,
    headheight=12pt,
    headsep=25pt,
    footskip=30pt
}

% ------------------------------------------------------------------------------

\begin{document}

% ------------------------------------------------------------------------------
% Cover Page and ToC
% ------------------------------------------------------------------------------

\title{ \normalsize \textsc{}
\\ [2.0cm]
\HRule{1.5pt} \\
\LARGE \textbf{\uppercase{CheatSheet ANDA}
\HRule{2.0pt} \\ [0.6cm] \LARGE{Formulas y resultados sobre Analisis de Datos} \vspace*{10\baselineskip}}
}
\date{}
\author{\textbf{Autor} \\
    Juan Horrillo Crespo \\
    Universidad de Valladolid \\
    \date{\today}
}

\maketitle
\newpage

\tableofcontents
\newpage

% ------------------------------------------------------------------------------

\section{Análisis por Componentes Principales}

\begin{itemize}
    \item Matriz de datos:
          \[X_{n \times k} = \begin{bmatrix}x_{11} & x_{12} & \cdots & x_{1k} \\ x_{21} & x_{22} & \cdots & x_{2k} \\ \vdots & \vdots & \ddots & \vdots \\ x_{n1} & x_{n2} & \cdots & x_{nk}\end{bmatrix}\]
    \item Individuo: Punto de $\mathbb{R}^k$
    \item Variable: Punto de $\mathbb{R}^n$
    \item Inercia Total: Suma de la distancia cuadrada de las observaciones al centro de gravedad G (lo situamos en 0)
          \[I_t = \sum_{i=1}^{n}d^2(x_i, G) = \sum_{i=1}^{n}(x_i-G)'(x_i-G) = \sum_{i=1}^{n}\left\lVert x_i \right\rVert^2 = tr(XX') = tr(X'X)\]
    \item Inercia explicada por $u$: Inercia explicada por el resumen unidimensional de los datos proyectados sobre $u$.
    \item $I_t = \sum_{i=1}^{k}I_{u_i}$
    \item El mayor autovector $u_1$ de $X'X$ maximiza la inercia recogida de entre cualquier otro vector.
    \item El siguiente mayor autovector $u_2$ de $X'X$ es el que recoge más inercia por detras de $u_2$.
    \item Sean $u_1, \dots, u_k$ los autovectores de $X'X$ entonces se verifica que $I_{u_1} \geq \dots \geq I_{u_k}$
    \item Descomposición en valores singulares: Sea $M$ matriz $k\times k$ simétrica definida no negativa. Entonces existe una matriz $U$ ortonormal tal que
          \[U'MU = \Delta\]
          con $\Delta = diag(\lambda_1, \dots, \lambda_k)$ y $\lambda_1 \geq \dots \geq \lambda_k \geq 0$.
    \item En nuestro caso $M = X'X$, por tanto $U$ esta formada por los autovectores que estamos buscando y $\lambda_i$ son los autovalores correspondientes.
    \item Como $tr(M) = tr(\Delta)$ se deduce que $I_t = tr(X'X) = tr(\Delta) = \sum_{i=1}^{k}\lambda_i$. O lo que es lo mismo, $I_{u_i} = \lambda_i$.
    \item Componente principal: Se define como componente principal de la siguiente forma \[F_\alpha = u_{1\alpha}X_1 + \dots + u_{k\alpha}X_k\]
          \newpage
    \item La proyección de un individuo $i$ sobre el eje $\alpha$ sería $F_{i\alpha} = x_i'u_\alpha$.
    \item Analisis Normado: Permite realizar analisis de forma que lo que se descubra no sea la diferencia entre las varianzas de las variables. Entre otras cosas, es util para comparar mediciones de distintas unidades.
    \item Sea $X^*$ una matriz de datos estandarizada, entonces la matriz $\frac{1}{n-1}(X^*)'X^*$ contiene no las covarianzas sino las correlaciones muestrales.
    \item Inercia acumulada: $I.a._s = \sum_{i=1}^{s}I_i$
    \item Porcentaje de inercia acumulada: $100\frac{\sum_{i=1}^{s}I_i}{\sum_{i=1}^{k}I_i} = 100\frac{I.c_s}{I_t}$
    \item Algunos criterios de retención de componentes:
          \begin{itemize}
              \item Considerar el porcentaje de inercia acumulado y decidir retener un numero de componentes s que haga que la inercia explicada supere un determinado porcentaje.
              \item Extraer aquellas componentes cuya inercia explicada supere el promedio de los autovalores (en el caso normado habitual esto es equivalente a extraer las componentes con autovalores mayores que 1).
              \item Construir el denominado "scree plot" en el que se representa en el eje de ordenadas el número del autovalor y en el de abcisas el autovalor en si y buscar el codo del gráfico.
          \end{itemize}
    \item Contribuciones absolutas (a la inercia explicada por cada eje): Se definen como $c.a.(i,\alpha)=\frac{(x_i'u_\alpha)^2}{(n-1)\lambda_\alpha}$. Nos dice lo que ha contribuido el individuo $i$ a la definición del eje $\alpha$.
    \item $\sum_{i=1}^{n}c.a.(i,\alpha)=1$ por lo que si un punto tiene contribuciones mucho más altas que el resto podemos dudar de la estabilidad del eje, ya que puede estar excesivamente condicionado por ese punto.
    \item Contribuciones relativas (cosenos cuadrados): Se definen como $c.r.(i,\alpha)=\cos^2(i,\alpha)=\frac{(x_i'u_\alpha)^2}{d^2(i,G)}$. Nos dicen como de cerca (o lejos) esta un punto de cada eje. Sirve para saber si un punto está mejor o peor representado en un eje o conjunto de ejes.
    \item $\sum_{\alpha=1}^{k}\cos^2(i,\alpha)=1$
\end{itemize}

\newpage

% ------------------------------------------------------------------------------
% Reference and Cited Works
% ------------------------------------------------------------------------------

\begin{thebibliography}{9}
    \bibitem{misc}
    Miguel Alejandro Fernández, \\ \textit{Diapositivas Análisis por Componentes Principales}. \\ Universidad de Valladolid, 2024.
\end{thebibliography}

% \bibliographystyle{IEEEtran}
% \bibliography{References.bib}

% ------------------------------------------------------------------------------

\end{document}
