\documentclass{article}

\usepackage{amsmath, amsthm, amssymb, amsfonts}
\usepackage{thmtools}
\usepackage{graphicx}
\usepackage{setspace}
\usepackage{geometry}
\usepackage{float}
\usepackage{hyperref}
\usepackage[utf8]{inputenc}
\usepackage[english]{babel}
\usepackage{framed}
\usepackage[dvipsnames]{xcolor}
\usepackage{tcolorbox}

\colorlet{LightGray}{White!90!Periwinkle}
\colorlet{LightOrange}{Orange!15}
\colorlet{LightGreen}{Green!15}

\newcommand{\HRule}[1]{\rule{\linewidth}{#1}}

\declaretheoremstyle[name=Teorema,]{thmsty}
\declaretheorem[style=thmsty,numberwithin=section]{theorem}
\tcolorboxenvironment{theorem}{colback=LightGray}

\declaretheoremstyle[name=Ejercicio,]{prosty}
\declaretheorem[style=prosty,numberlike=theorem]{exercise}
\tcolorboxenvironment{exercise}{colback=LightOrange}

\declaretheoremstyle[name=Demostración,]{prcpsty}
\declaretheorem[style=prcpsty,numberlike=theorem]{proofs}
\tcolorboxenvironment{proofs}{colback=LightGreen}

\setstretch{1.2}
\geometry{
    textheight=9in,
    textwidth=5.5in,
    top=1in,
    headheight=12pt,
    headsep=25pt,
    footskip=30pt
}

% ------------------------------------------------------------------------------

\begin{document}

% ------------------------------------------------------------------------------
% Cover Page and ToC
% ------------------------------------------------------------------------------

\title{ \normalsize \textsc{}
		\\ [2.0cm]
		\HRule{1.5pt} \\
		\LARGE \textbf{\uppercase{Template Title}
		\HRule{2.0pt} \\ [0.6cm] \LARGE{Subtitle} \vspace*{10\baselineskip}}
		}

\author{\textbf{Author} \\ 
		Victor Elvira Fernandez, Tomás Ruiz Rojo, Juan Horrillo Crespo \\
		Universidad de Valladolid \\
		\date{\today}}

\maketitle
\newpage

\begin{center}
    \Huge \textbf{AVISO}
\end{center}

Estos apuntes fueron creados de forma voluntaria por un grupo de estudiantes, invirtiendo tiempo, dedicación y esfuerzo para ofrecer información útil a la comunidad. Apreciamos cualquier apoyo que se nos quiera brindar, ya que nos ayuda a continuar con futuros proyectos de este tipo. \\

Si deseas colaborar en esta clase de proyectos puedes contactarnos y unirte o invitarnos a unas ricas patatas 5 salsas por el siguiente enlace:

\vfil

\begin{center}
    \href{https://www.buymeacoffee.com/ApuntesINdat}{\LARGE \textbf{Buy Me a Patatas 5 Salsas}}
    \href{https://www.buymeacoffee.com/ApuntesINdat}{https://www.buymeacoffee.com/ApuntesINdat}
\end{center}

\begin{itemize}
    \item \href{mailto:juan.horrillo22@estudiantes.uva.es}{Mail Juan Horrillo}
    \item \href{mailto:victor.elvira22@estudiantes.uva.es}{Mail Victor Elvira}
    \item \href{mailto:tomas.rojo22@estudiantes.uva.es}{Mail Tomás Rojo}
\end{itemize}

\vfil

Si has colaborado de cualquier forma te agradecemos enormemente.

\tableofcontents
\newpage

% ------------------------------------------------------------------------------
\section{Primera sección}

Vel consectetur corrupti eum unde deserunt et asperiores harum. Dolorem fugiat cum sequi culpa. Nemo dignissimos et hic qui et non nemo. Iusto id quia debitis vitae. Expedita dolorem inventore eos facere reprehenderit quasi.

Doloremque fugit non veniam recusandae perferendis. Ipsum ad velit inventore et vitae exercitationem ipsam explicabo. Qui sapiente non eaque expedita deserunt possimus iusto iste. Odit occaecati ducimus harum culpa. Est eius omnis consequatur ut earum maxime dolore natus. Nostrum non molestiae voluptas nisi quisquam fugiat dolor.

Aliquam sed ipsa natus nam laudantium consequuntur. Repellat cumque vel voluptatem enim similique blanditiis. Et quod qui quasi doloribus quia adipisci similique. Eveniet velit ducimus odit molestiae magni. Sit ex rem et et quibusdam. Minus et reprehenderit et architecto et reprehenderit ut ullam.

Optio quia molestiae cumque et. Porro magnam quia veritatis illum temporibus officiis. Laborum quod quis impedit ducimus deleniti. Officia omnis aspernatur quis est odit ex labore. Eaque non eveniet vel. Deserunt tempora aliquam pariatur quo.

Quas dignissimos excepturi similique nemo. Omnis sint dignissimos nobis mollitia commodi. In odio fugiat nesciunt. Tenetur odit tempora praesentium alias delectus soluta. Dicta et commodi nostrum facilis inventore sunt nisi ex.

\section{Teorema, Demsotración y Ejemplo}

\begin{theorem}
    Teorema de ejemplo
\end{theorem}

\begin{proofs}
    Demsotración de ejemplo
\end{proofs}

\begin{exercise}
    Ejemplo de \textit{ejemplo}$\dots$
\end{exercise}

\section{Matemáticas}

\begin{itemize}
    \item Matemáticas en línea: \\
    Puedes escribir entradas matemáticas en linea como $\phi \sim N(0,1)$ y seguir escribiendo$\dots$
    \item Matemátricas en ``bloque'': \\
    Tambien puedes escribir entradas matemáticas en bloque como 
    \[
        \phi \sim N(0,1)
    \]
    y seguir escribiendo$\dots$
\end{itemize}

% ------------------------------------------------------------------------------
% Reference and Cited Works
% ------------------------------------------------------------------------------

\begin{thebibliography}{9}
    \bibitem{manuscritos}
    Juan Camilo Yepes Borrero, \\ \textit{Apuntes INFE2 ...}. \\ Universidad de Valladolid 2024.
    \bibitem{apuntes}
    Yolanda Larriba González, \\ \textit{Apuntes INFE2 ...}. \\ Universidad de Valladolid 2023.
\end{thebibliography}

% ------------------------------------------------------------------------------

\end{document}
