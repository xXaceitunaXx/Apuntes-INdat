\section{Distribución multinomial}


Dado un experimento aleatorio con $k$ resultados posibles, de tal modo que la probabilidad de cada resultado $p_1, p_2, \dots, p_k$ se mantiene constante,
un vector aleatorio $(X_1, X_2, \dots, X_k)$ sigue distribución multinomial si cada $X_{i}$ representa el número de veces que ocurre el resultado $i$-esimo
en $n$ repeticiones independientes del experimento. \\
Dados $0 \leq x_1, x_2, \dots, x_k \leq n$ y $0 \leq p_1, p_2, \dots, p_k \leq 1$ con $\sum_{i = 1}^{k}x_i = n$ y $\sum_{i = 1}^{k}p_i = 1$, si 
$(X_1, X_2, \dots, X_k)$ sigue distribución multinomial, tenemos

\[ P(X_1 = x_1, X_2 = x_2,\dots,X_k = x_k) = \frac{n!}{x_{1}!x_{2}!\cdots x_{k}!}p_{1}^{x_1}p_{2}^{x_2}\cdots p_{k}^{x_k}\]

Propiedades:
\begin{itemize}
    \item Todas las distribuciones marginales de un vector multinomial son multinomiales (binomiales en su caso)
    \item Todas las distribuciones condicionales de un vector multinomial son multinomiales (binomiales en su caso)
\end{itemize}
Por ejemplo, podemos decir que, si $(X_1,\dots,X_k)\thicksim\mathcal{M}(n;p_1,\dots,p_k)$ entonces:
\begin{itemize}
    \item $X_i\thicksim b(n, p_i),\quad i = 1,\dots,k$
    \item $X_1 + \cdots + X_i\thicksim b(n, p_1 + \cdots + p_k), \quad i = 1, \dots, k$
\end{itemize}
